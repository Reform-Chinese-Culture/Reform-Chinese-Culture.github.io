\documentclass[12pt]{article}
% \usepackage{assets/sty/reformstyle-mobile}
\usepackage{assets/sty/reformstyle}
\usepackage[notes]{biblatex-chicago}
\addbibresource{references.bib}

\begin{document}

\title{Meritocracy yet to be defeated}
\subtitle{A rebuke against Michael Sandel}  % Using author field for subtitle
\author{URCC}
\date{August 24, 2025}
\maketitle

\begin{preamble}
Michael Sandel wrote on the \emph{Tyranny of Merit}.\footcite{sandelTyrannyMeritWhats2020} This article first summarizes his arguments in bullet points. I then introduces my brief rebuttals, followed by other academic sources which might be of interest.
\end{preamble}

\section{Reading Notes}

\subsection{The Rise of Meritocracy}
\begin{enumerate}
    \item Old Catholic tradition of salvation by work: there was a connection between work and deserving reward.
    \item Protestant Reformation sustained the work ethic.
    \begin{enumerate}
        \item Calvinism, by predeterminism, inevitably encouraged this mindset.
    \end{enumerate}
    \item In modern U.S. times:
    \begin{enumerate}
        \item U.S. presidents increasingly use language such as ``the \ldots that you \emph{deserve},'' or ``on the right side of history.''
        \item Martin Luther King, Jr.: ``[T]he arc of the moral universe is long, but it bends toward justice.''
        \item Ivy League graduates increasingly fill cabinet appointments (e.g., Obama's well-credentialed cabinet ministers), leading to apathy against the white working class; hence the Trumpian cultural backlash.
    \end{enumerate}
\end{enumerate}

\subsection{Economic Theories of Merit}
\begin{enumerate}
    \item Hayek's free-market liberalism: reward is not based on merit, but on the manipulation of the market and reflective of market values (\emph{wants} or \emph{desires}).
    \item Rawls' welfare-state liberalism: accepts reward based on market values instead of merit, but emphasizes redistribution because the ability to accrue market values is unearned. 
    \begin{itemize}
        \item Example: born disabled, or born to excel in martial arts in a society that only values literary skills.
        \item Redistribution, however, makes poor people receiving government transfers feel unworthy.
    \end{itemize}
    \item Frank Knight's critique of equating market values with merit/deservingness: e.g., inventing an addictive drug provides for market wants and accrues value, but is not deserving. 
    \begin{itemize}
        \item In fact, the people satisfying undeserving market wants (drugs, gambling) may be the very ones creating them.
    \end{itemize}
    \item Luck egalitarianism: defends inequalities that arise from effort and choice. This highlights a point of convergence with free-market liberalism.
\end{enumerate}

Sandel claims that all theories in (1)--(4) lead to meritocratic hubris.

\subsection{The Harms of Meritocracy}
\begin{enumerate}
    \item Harmful for poor individuals: psychological effects of being told their ``non-deservingness'' is due to forces outside their control.
    \item Harmful for society: meritocratic hubris from top earners and the privileged.
    \item No dignity of work: even when poor Americans receive payouts, they suffer blows to self-esteem when dependent on government help.
\end{enumerate}

\section{My Comments}

Sandel correctly observes the woes of the white working class in the U.S. However, he incorrectly attributed it to the rise of meritocracy. And he didn't prove why meritocracy is harmful in the first place.

Even three-quarters into the book, Sandel seems merely to be ranting about the rise of meritocracy and expecting readers to decide whether it is good or bad. Furthermore, the examples he constantly cites centres on the dissatisfaction of the white working class. What about Black and Latino Americans? Asian Americans? American Jews? These groups cannot be ignored, especially when affirmative action increasingly puts them under the spotlight.

To give him some fair credit, Sandel raised a good question about the white working class (WWC). He correctly attributed many of their woes to globalization and an economy mercilessly leaving people without technical skills behind. But what is the diagnosis? He argues that meritocracy is rising and is to blame for the woes of the white working class whose jobs are disappearing into the fast pace of globalization.

His main argument against meritocracy is psychological. But the argument that meritocracy causes poor Americans psychological harm is too intangible to prove and borders on socialism. First, why does wealth always lead to hubris? What about Bill Gates and others active in philanthropy? What about \emph{noblesse oblige}? Even the meritocratic religions Sandel cites (e.g., Catholicism) emphasize charity. A survey (e.g., Pew) of top 10\% attitudes toward the poor would be more concrete. Second, why is receiving help necessarily harmful for self-esteem? Why can't it instead rebuild dignity and enable a middle-class life under globalization? If receiving redistribution is always harmful, Sandel seems to argue for eliminating wealth differences altogether. Only then would redistribution be unnecessary. Under this stringent constraint of Sandel, it's hard to see any other alternatives than socialism that do not harm the dignity of work and the self-esteem of the poor white working class. 

Even if Sandel does not explicitly endorse socialism, he underplays how market economies invigorate innovation to level up the living standards for all. The proper test would be a controlled experiment that asks: what helps the bottom 20\% more in these two scenarios?
\begin{itemize}
    \item Economy (A): income based on contributions, satisfying wants, or merit.
    \item Economy (B): welfare market economy (status quo).
\end{itemize}

Such a question is worth investigating: even some may argue that the US market economy has been more inequal in wealth, the economy could lift \emph{everyone} up so fast that even the bottom 20\% got a huge living standards boost. For example, Economy (B) could level up the living standards by 500\% in 10 years, giving a 1000\% boost to the top 20\% and a 200\% boost to the bottom 20\%. At the same time, because Economy (B) has no incentives for engineers for better fridges, cars, trains to innovate, it uniformly levels up everyone by 100\%, less than the 200\% in (A). 

To be sure, Economy (A) has never existed in real life. So whatever comparisons we're making only exist in theory. Sandel didn't do anything to prove why Economy (A), in theory, is better. Even if Sandel were to succeed at proving the benefits of (A), there's another question that's outstanding: how to reach (A)?

Attempts to establish an Economy (A) in the Soviet Union, China, North Korea, Cuba, and some parts of Israel have either failed or perverted to be something else. But it is simply dangerous for Sandel to either ignore this argument, or to argue for the establishment of an Economy (A). This is important: because there've been so many failed attempts to establish an alternative to the market economy, any theorie is dubious if it argues for its benefit \emph{before} even discussing the establishing process itself. You can't just advocate that Mars is a better planet to live on\textemdash{}you have to establish the feasibility of moving to Mars first.

In sum, Sandel asked a great question about the poor whites (no other poor Americans are specifically discussed in this book), while giving an answer to neither the poor whites nor the poor Americans as a whole. Reading this book against meritocracy ultimately makes me want to doubling down on the meritocratic idea of equality of opportunity for all. I now believe even more in the \emph{merit} of meritocracy.

\section{For Further Discussion}

Plattner\footcite{plattnerTyrannyMeritWhats2021} and Bridges\footcite{bridgesTYRANNYRACEBLINDNESS2021} offer further insights.

\section{References}

\printbibliography[heading=none]

\reformcopyright{2025}

\end{document}