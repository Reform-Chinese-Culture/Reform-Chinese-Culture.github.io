\documentclass[12pt]{article}
% \usepackage{assets/sty/reformstyle-mobile}
\usepackage{assets/sty/reformstyle}
\usepackage{hyperref}
\usepackage{ellipsis}[Chicago]
\usepackage{csquotes}
% \usepackage{lips}[Chicago]

\begin{document}

\doctitle{Letter to the Editor: \textit{Abundance for the 99 Percent} in \textit{Jacobin}}

\docsubtitle{Why centrism still works}

% \begin{preamble}
	The book review \textit{Abundance for the 99 Percent}\footnote{\href{https://jacobin.com/2025/08/klein-thompson-abundance-liberalism-socialism}{https://jacobin.com/2025/08/klein-thompson-abundance-liberalism-socialism}} was published in \textit{Jacobin} on August 2, 2025. In \textit{Jacobin} review, Huber, Phillips, and Stafford review the book \textit{Abundance} written by Ezra Klein and Derek Thompson.\footnote{\textit{Abundance} by Ezra Klein and Derek Thompson (Simon \& Schuster, 2025)}

	In this critique of the book review, and in defense of the true intentions of the original book, I perform a criticism technique that I call \textit{in situ}.\footnote{Here, I perform a criticizm technique called ``\textit{in situ}.'' Rather than focusing on arguing for the opposing side of the material, \textit{in situ} mostly takes apart material without rebuilding the alternative. Furthermore, \textit{in situ} uniquely focuses on the symbolic logic of the original argument. This means that the \textit{in situ} analysis can be extrapolated to suit not only the particular set of evidence, but also a general set of patterns. Therefore, although some may argue that the \textit{in situ} technique is facile and not constructive, it nevertheless serves as a powerful critique. Instead of attempting to definitively prove that the opposing argument is true, \textit{in situ} seeks to provokes more intellectual engagement by pointing out the weakness of the original arguments.}
% \end{preamble}

The main unique contribution of the \textit{Jacobin} book review is that Abundance agenda and socialism go hand-by-hand. The book review, correctly, provides deeply analysis and background information on Neoliberalism, Jeffersonianism versus Hamiltonianism, mentioning of the ``Groups'' of NGOs, lawyers, and NIMBYists chocking supply. However, there is a dearth of mechanism in arguing the unique contribution of review\textemdash{} Abundance needs socialism, and socialism needs Abundance.

In this letter, I analyze each part of the book review article one by one with \textit{in situ}\footnote{See \textit{footnote 3}.} critiques, proving why the central thesis is nothing more than a propaganda tool of co-option.

In the beginning of the book review, Huber et al.\ makes clear that their thesis is that abundance cannot happen without socialism, and vice versa.\footnote{para. 6. \textit{So as we articulate our own critiques of the book \dots}} Following the thesis, the review have roughly the following subheadings\textemdash{} all but only the last one are irrelvant. (1) Is Abundance a rehashing of Neoliberalism? (2) Jefferson or Hamilton? (3) The anti-abundance ``Groups'' (of NGOs, lawyers, etc)? (4) Bureaucracy? (5) NGO-industrial complex?\footnote{These subheadings are paraphrased for brevity.} And finally, the one and only  subheaing relevant to the central thesis: (6) Abundance Needs Socialism and Socialism Needs Abundance.

In the first few subheadings (1)-(2), Huber et al.\ unjustifiably attempts to recast the book in the light of socialism. In the discussion of topic (1), Neoliberalism, the \textit{Jacobin} writers made a few mistakes in mistakenly making the readers believe that the book has socialistic characteristics. First, they made the book sound more state-centric than it is. Only the extreme examples are cited by Huber et al.\ where \textit{Abundance} supports that the state steps in. For example, Huber et al.\ correctly characterizes Klein and Thompson as pro-government in a climate crisis, market failures,\footnote{para. 10.} industrial policy on DARPA, and mRNA research.\footnote{para. 12.} Although the book reviewers did not explicitly call these measures socialistic, it would be a mistake to hint at socialism for later. In fact, almost all Western democracies would support these measures during extreme times\textemdash{} most of them are not social democracies like Scandinavian countries. 

The main contribution of the Klein and Thompson book is more about the ordinary times\textemdash{} we don't build high-speed rail with the same sense of urgency as making COVID vaccines; we don't build housing with the same sense of urgency as during a climate crisis. It is unclear why these examples are even relevant in support of the forged view that Klein and Thompson support state-centrism in all cases. Well, they don't. They, like almost all reasonable centrist governments, support state intervention and state-centric policies only in extreme cases.

Even later in para. 15\footnote{para. 15. \textit{``We looked into it,'' they write, ``and it turns out that all those countries also have governments, so the problem cannot be government.''}}, they try to co-opt Klein and Thompson on the discussion of the costs of building high-speed rail. The discussion concerns the low rail-building costs in countries like Portugal and Germany compared to the high costs in the US.\@ While Klein and Thompson says that all these countries have governments, the book reviewers co-opted that logic and said that the government can be the problem. However, it is clear that the government has many flavours, some big, some small. It is unclear why the simple fact that all countries have governments is a reason why the US government is not at fault for having high rail-building costs.

Unfortunately, this litany of overreaching arguments continue onto the second topic, where the \textit{Jacobin} writers argued the Klein and Thompson intended to (2) rediscover the Hamiltonian tradition. Huber et al.\ argued that Klein and Thompson are Hamiltonians because they want a strong government to counter big corporations, to ``keep the power of business in check'';\footnote{para. 26. \textit{Yet these reformers shared the anti-monopolists’ concern about concentrated corporate power \dots}} rathern than being the Jeffersonians who want to have small individual companies instead of big corporations. They then suggest that a government is not enough\textemdash{} it has to be a socialist government that ``builds housing, energy, and other key infrastructure.\footnote{para. 32. \textit{As early as 2022, Klein dubbed this vision a “liberalism that builds” — that is, builds housing, energy, and other key infrastructure \dots}}

On topics (3) and (4), the \textit{Jacobin} writers first restated Klein and Thompson's proposition that the ``Groups'' of NGOs, lawyers, and NIMBYists are anti-abundance, and that they are a symptom of pendulum imbalance. However, they claim that the origin of such ``Groups'' is not sufficiently answered by the simple pendulum-going-too-far analogy. What's the real origin? Huber et al.\ offers a socialist explaination. Or is it really? Huber et al.\ writes,

% \begin{quotation}
	The socialist defense of the public sector is not a defense of government per se, but rather a defense of self-governance — that is, of democracy.\footnote{para. 56.}
% \end{quotation}

Huber et al.\ argues that capitalists are not held accountable by their actions when deciding whether ``a certain good or service will or will not be produced.''\footnote{para. 54.} Because of this, the only viable alternative is a democratic government with a mandate from the collective public. However, Huber et al.\ concedes that the government is no better if it falls under the whims of an authoritarian party with no accountability. However, it is unclear why Huber's idealistic socialism will bring such accountability to the status quo\textemdash{}the book reviewers themselves conceded the USSR case, where a corrupt authoritarian party operates on the banner of socialism. Not to mention China, North Korea, Cuba, Venezuela, etc. Even if Huber et al.\ contends that the USSR was not real socialism, they still have to meet the tall order of why a future implementation of socialism will not fall into the same trap.

Furthermore, it is unclear why Huber et al.\ would attribute the notion of democratic accountability of the government to only socialism. In fact, Canada in 2006 passed the \textit{Federal Accountability Act} to cap personal donations (read: donations from rich people) and ban union and corporate donations.\footnote{\href{https://laws-lois.justice.gc.ca/eng/acts/f-5.5/}{https://laws-lois.justice.gc.ca/eng/acts/f-5.5/}} To be sure, it was Stephen Harper, the Conservative Prime Minister, not a soft-hearted left-wing politician, who enacted the law. The UK, in a presumably well-intentioned attempt to expand accountability, gave 16-year-olds the right to vote.\footnote{\href{https://www.gov.uk/government/news/16-year-olds-to-be-given-right-to-vote-through-seismic-government-election-reforms}{https://www.gov.uk/government/news/16-year-olds-to-be-given-right-to-vote-through-seismic-government-election-reforms}} It seems that what Huber et al.\ is proposing is not a unique contribution of socialism\textemdash{}if ensuring accountability of government makes you a socialist, then most people in North America and Western Europe are socialists already.

Huber et al.\ then elaborated,

% \begin{quotation}
	By the 1970s, the Left had developed internal critiques of both welfare-state and trade-union bureaucracies — critiques that owe a great deal to these earlier attempts to theorize Stalinism. In both cases, even though western welfare states and trade unions were democratic, institutional arrangements had emerged that increasingly sheltered actors within them from accountability. These actors, in turn, began to act in their own interests, distinct from any mandate they had been given.\footnote{para. 59.}
% \end{quotation}

It seems that Huber et al.\ is meditating on the left's own failures. What follows is unclear. Huber et al.\ didn't answer the question: should we go one step more to the left to achieve full-throated socialism? Or should we, as Klein and Thompson suggested, swing the pendulum back to the centre? Huber et al.\ posits a question, but doesn't give an answer.

In topic (5), the book reviewers briefly mentioned the NGO-industrial complex, but didn't meaningfully connect it to their central thesis about the necessity of socialism. Huber et al.\ pointed out, rightfully so, examples such as NGOs and unions unjustifiably putting up barriers using adversarial legalism against clean nuclear power.\footnote{para. 67.} This closely echos the previous para. 47 where Huber et al.\ cited the example of the ``political capture of progressive politics by the Groups'' in nuclear deals.\footnote{para. 47.} As one last attempt to justify socialism, they cited Vaheesan's argument that market failures are the main cause of slow deployment of nuclear energy: firms are unwilling to take on enormous up-front financial burdens to build nuclear infrastructure.\footnote{para. 69.}

There are two critiques on this. First, the fact that NGOs are capturing the government is not necessarily a good argument to have stronger Hamiltonian government. In fact, one of the best ways to reduce the capture effect is to break the NGOs and many other corporate entities into small Jeffersonian pieces. If no one single group wields the authoritarian power, the society will be more democratic. However, as Huber et al.\ themselves are likely to admit, there seems to be a limit on how accountable the government can be. We have tried proportional representation (PR) which can lead to the rising of the far-right. We have tried first-past-the-post (FPTP) which many argue unjustifiably waste votes and can be gerrymandered to manipulate elections. Some countries are trying the hybrid of PR and FPTP with inconclusive results. Therefore, if the maximum accountability from the side of the government has been reached, it is up to the audacious people to create diverse factions of their own to speak freely. Huber et al.\ example could instead be read as an inspiration to be more Jeffersonian.

Second, regarding the market failures. Even in the non-socialist status quo, most countries would have a nuclear power plan supported by the state. Reasonably, it would require troops to guard the nuclear power plants; it requires substantial resources that the state can provide. Furthermore, even under the Neoliberal status quo, economics research has shown that the economics of scale incentivizes larger entities to operate power plants, e.g., the government, a crown corporation, or a large private company. That is because after achieving massive economics of scale, the average cost of power per watt would be cheaper eventually. However, this is likely to be symmetrical on both sides of the status quo and socialism. Building a behemoth entity, private or public, is inherently difficult. It is unclear why the government would build one faster than private companies incentivized by sweet bulk profits.

Finally, the last subheading (6) is the only one that is relevant to the central thesis of the book review. Huber et al.\ provide three reasons for why Abundance needs socialism. However, all three reasons are predicated on one misconception\textemdash{} that there does exist a dominating political class which necessitates the working class-led socialism. They pre-suppose that the capitalists will artifically constrain supply; that capitalists hold too much power; that working class can provide solution. When socialists chant ``Workers of the world, unite!'', what they ignore is that the working class is only more fractured today than ever. Todays' 

men, republicans

assume best buddies


left, differentiate. Failed project
\reformcopyright{2025}

\end{document}