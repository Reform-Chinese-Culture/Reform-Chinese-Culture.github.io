\documentclass[12pt]{article}
% \usepackage{assets/sty/reformstyle-mobile}
\usepackage{assets/sty/reformstyle}
\usepackage{hyperref}
\usepackage{ellipsis}[Chicago]
\usepackage{csquotes}

\usepackage[english]{babel}
\usepackage[notes]{biblatex-chicago}
\addbibresource{references.bib}

\begin{document}

\title{Reform Chinese Culture}
\subtitle{For All Voices, In All Places}
\author{Union for Reform Chinese Culture (URCC)}

\maketitle

\section{What I Write}
\href{202508Abundance.html}{Letter to the Editor: \textit{Abundance for the 99 Percent} in \textit{Jacobin}---Why centrism still works} August 4, 2025

\href{202508DebateOnPhilosophyPaulGrahamPG.html}{A Debate on How to Do
Philosophy---Paul Graham vs. James Somers: Is philosophy more than a word salad?} August 5, 2025

\section{What I Browse}

\enquote{You're probably using the wrong dictionary.}\footcite{somerYoureProbablyUsing} Arguing for a return of the original Webster's edition of dictionary which uses livelier, more classical, and context-specific language

\enquote{Google Scholar.}\footcite{googleGoogleScholar} Combined with your individual scholarship library subscription, a powerful search engine.

\section{Mission}

The Union for Reform Chinese Culture (URCC) seeks to reform Chinese culture to accommodate progressive and democratic values using radical centrism,\footnote{\href{https://en.wikipedia.org/wiki/Radical\_centrism}{https://en.wikipedia.org/wiki/Radical\_centrism}} with reasoning based on the classical and contemporary Chinese cultural canon, to benefit the Chinese diaspora and all people of Chinese culture.

% \section{What is Reform Chinese Culture?}

% In his famous play \textit{Teahouse} (《茶馆》),\footnote{茶馆 (Cha2 Guan3)} the playwright \textit{Lao She} (老舍) wrote about the omnipresence of the ``Do Not Discuss National Matters'' (莫谈国事)\footnote{莫谈国事 (Mo4 Tan2 Guo2 Shi4)} signage on the walls of teahouses in Beijing, China. The scene was set in 1898, in the decade of 1890. But what's not to be discussed? What national matters?

% Not to discuss the three years before 1898, during 1894--1895, when the Qing-Dynasty China lost to Japan in the First Sino-Japanese War (甲午战争)\footnote{甲午战争 (Jia3 Wu3 Zhan4 Zheng1)} against Japan's superior military technology, ceding Taiwan to Japan.

% Not to discuss the three decades before 1898, from 1861 to 1895, when the Self-Strengthening Movement (洋务运动)\footnote{洋务运动 (Yang2 Wu4 Yun4 Dong4)} strived to bring the military technology and manufacturing know-how to the Qing Dynasty, while largely ignoring the Western Culture, chanting ``Chinese Learning as Substance, Western Learning for Application.''\footnote{中学为体,西学为用 (Zhong1 Xue2 Wei2 Ti3, Xi1 Xue2 Wei2 Yong4). \href{https://zh.wikipedia.org/wiki/洋务运动}{Wikipedia} recommends this source for further reading: p.g. 253, 黄仁宇. 《中国大历史》. 北京: 三联书店. 1997. ISBN 9787108010360}

% Not to discuss about half a century before 1898, from 1840 to 1960, when the Qing-Dynasty China lost two Opium Wars,\footnote{鸦片战争 (Ya1 Pian4 Zhan4 Zheng1)} ceding Hong Kong to the British Empire and signing away control over trade and tariffs to foreign powers.

% Not to discuss, in the same century of 1898, from 1881 to 1884, over 17,000 Chinese immigrants were in Canada to build the Canadian Pacific Railway.\footnote{\href{https://www.canada.ca/en/canadian-heritage/campaigns/asian-heritage-month/important-events.html}{https://www.canada.ca/en/canadian-heritage/campaigns/asian-heritage-month/important-events.html}}

% Not to discuss the thirteen years before 1898, in 1885, when the Canadian government imposed the Chinese Head Tax and banned all Chinese immigrants from voting in federal elections. On July 1, 1923, Dominion Day, the Canadian government passed the Chinese Exclusion Act.\footnote{\href{https://www.canada.ca/en/canadian-heritage/campaigns/asian-heritage-month/important-events.html}{https://www.canada.ca/en/canadian-heritage/campaigns/asian-heritage-month/important-events.html}}

% Back to the year of 1898 when the play was set, not to discuss the Hundred Days' Reform (戊戌变法)\footnote{戊戌变法 (Wu4 Xu1 Bian4 Fa3)} aimed to incorporate Capitalistic characteristics, expand democratic venues, and abolish the imperial examination system.

% But all these national matters, national history, WE NEED TO DISCUSS!

% For only 20 years after 1898, in 1918, \textit{Lu Xun} (鲁迅) wrote \textit{Diary of a Madman} (狂人日记),\footnote{狂人日记 (Kuang2 Ren2 Ri4 Ji4)} categorically condemning exploiting people under the guise of Confucian virtues.\footnote{仁义道德 (Ren2 Yi4 Dao4 De2)}

% For only 21 years after 1898, in 1919, the May Fourth Movement (五四运动)\footnote{五四运动 (Wu3 Si4 Yun4 Dong4)} united students and intellectuals from every socio-political and intellectual factions to \textit{shi wei} (protest), \textit{ba ke} (cancel classes), \textit{ba shi} (cancel markets), \textit{ba gong} (cancel employment). The students declared that the situation was so dire that ``it cannot accommodate a quiet desk in a classroom anymore.''\footnote{华北之大,已经安放不得一张平静的书桌了 (it cannot accommodate a quiet desk in a classroom anymore) from \href{https://xsg.tsinghua.edu.cn/info/1006/1063.htm}{https://xsg.tsinghua.edu.cn/info/1006/1063.htm}}

% For only in the next 50 years, China paid a heavy price in the First and Second World Wars.

% For only 51 years after 1898, in 1949, the People's Republic of China was founded. For only 80 years after 1898, in 1978, China began its Reform and Opening-Up (改革开放).\footnote{改革开放 (Gai4 Ge2 Kai1 Fang4)}

% On the other side of the earth, in 1957, Douglas Jung became the first Chinese Canadian elected to the House of Commons, representing Vancouver Centre.\footnote{\href{https://www.canada.ca/en/canadian-heritage/campaigns/asian-heritage-month/important-events.html}{https://www.canada.ca/en/canadian-heritage/campaigns/asian-heritage-month/important-events.html}} In 2004, Raymond Chan became the first Chinese Canadian cabinet minister (albeit only a junior Minister of State).\footnote{\href{https://openparliament.ca/politicians/62/}{https://openparliament.ca/politicians/62/}} In 2006, Michael Chong became the first Chinese Canadian cabinet minister.\footnote{\href{https://michaelchong.ca/about-michael/}{https://michaelchong.ca/about-michael/}}

% But still, Do Not Discuss National Matters?

% About 30 years back, in 1871, the Jews were granted full citizenship in the unified Germany.

\section{URCC Manual of Style}

\begin{enumerate}
	\item The website formatting shall be based in large part on \href{https://edwardtufte.github.io/tufte-css/}{https://edwardtufte.github.io/tufte-css/}, which is Dave Liepmann's adaptation of Edward Tufte's style.
	\item The text formatting shall be loosely based on the latest Chicago Manual of Style (CMOS).\footnote{\href{https://www.chicagomanualofstyle.org/home.html}{https://www.chicagomanualofstyle.org/home.html}}
	\item Where appropriate, for the purpose of increasing readability, with the consideration of the overwhelming number of website citations, the CMOS citation style may be simplified.
	\item In quoting from a website, the first few words of a paragraph may be included while italicized, to reference the appropriate section of the website. This is to remedy the lack of page numbers and paragraph numbers on a website.
	\item If the context is enough to identify the exact location of the source on a website, the footnote or sidenote may be omitted. Users can use the browser's search function to locate the source quote.
\end{enumerate}

\section{Copyright and license of use}

All rights reserved.

\section{References}

\printbibliography[heading=none]

\reformcopyright{2025}

\end{document}