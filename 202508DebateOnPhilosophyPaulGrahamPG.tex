\documentclass[12pt]{article}
% \usepackage{assets/sty/reformstyle-mobile}
\usepackage{assets/sty/reformstyle}
\usepackage{hyperref}
\usepackage{ellipsis}[Chicago]
\usepackage{csquotes} 
% Use \enquote{} for quotes
% Use \begin{quotation} as a block quote

\begin{document}

\title{Is philosophy more than a word salad?}
\subtitle{Paul Graham vs. James Somers}  % Using author field for subtitle
\author{URCC}
\date{August 5, 2025}
\maketitle

\href{202508DebateOnPhilosophyPaulGrahamPG.pdf}{202508DebateOnPhilosophyPaulGrahamPG.pdf}

\begin{preamble}
 Paul Graham argued\footnote{Graham, Paul. Sept 2007. "How to Do Philosophy." \href{https://www.paulgraham.com/philosophy.html}{https://www.paulgraham.com/philosophy.html}} for a more result- and utility-oriented approach to studying philosophy, against philosophers who wrote  ``word salads'' without practical implications. He then proposed a test of utility as an integral part of revolutionizing \emph{how to do philosophy}.

 James Somers defended\footnote{Somers, James. n.d. \enquote{A Defense of Philosophy (against Paul Graham).} \href{https://jsomers.net/pg-philosophy.html}{https://jsomers.net/pg-philosophy.html}} the discipline of philosophy, especially against the "word salad" critique; instead, he argued for the appropriate usage of complex jargonic language in philosophy. Furthermore, he argued that philosophy meets the test of utility, as Graham proposed.

 This article aims to present both sides, and if possible, reconcile them.
\end{preamble}

\section{Graham against philosophy}

In the polemic against the status quo philosophy, Paul Graham cited \enquote{confusion of words} as a significant issue. \enquote{Do we have free will? Depends \emph{[sic]} what you mean by \enquote{free.} Do abstract ideas exist? Depends \emph{[sic]} what you mean by \enquote{exist.}} He then lauded Wittgenstein's approach to defining the language, but lamented that philosophy is still bogged down in essays of words of indeterminate meaning. He argues that philosophy is much like mathematics, albeit with a different medium. On the one hand is mathematics operating on precisely defined terms, and on the other is philosophy.

Graham laments that the discipline doesn't pursue \emph{useful} knowledge. Here, partly in parallel to mathematics, he argues that philosophy is disingenuous in claiming that philosophy should pursue knowledge of no practical use, being the supposedly more noble goal. He remarked that this shields the discipline from criticism when it doesn't bear useful fruit. Furthermore, he argues that Aristotle and his tutelage pursued it in the completely wrong direction:

\begin{quotation}
 [H]e sets as his goal in the Metaphysics the exploration of knowledge that has no practical use. Which means no alarms go off when he takes on grand but vaguely understood questions and ends up getting lost in a sea of words.
\end{quotation}

He said that the motive to pursue uselessness is different from the result of being useless. E.g., he cited number theory as a pursuit of useless knowledge that turns out to be quite useful. However, in claiming this, it is unclear why the intention to pursue useless knowledge will \emph{inevitably} lead to knowledge that is useless. Graham didn't explain how the intention will inevitably lead to the result. In fact, number theory is a counterexample to Graham's blanket statement. Furthermore, Graham's argument that the pursuit of uselessness shields philosophers is valid \emph{only when} philosophy is useless. Somers later attacks this point.

Graham also gives an interesting interpretation of the philosophy scholarship. He conceptualizes two kinds of people: (1) people who deem philosophy a waste of time will study other subjects; (2) people who can't tell the difference between justifiably difficult things (e.g., mathematics) and unjustifiably difficult things (e.g., philosophical word salad), as he writes:

\begin{quotation}
 Curiously, however, the works they produced continued to attract new readers. Traditional philosophy occupies a kind of singularity in this respect. If you write in an unclear way about big ideas, you produce something that seems tantalizingly attractive to inexperienced but intellectually ambitious students. Till one knows better, it's hard to distinguish something that's hard to understand because the writer was unclear in his own mind from something like a mathematical proof that's hard to understand because the ideas it represents are hard to understand. To someone who hasn't learned the difference, traditional philosophy seems extremely attractive: as hard (and therefore impressive) as math, yet broader in scope. That was what lured me in as a high school student.
\end{quotation}

While his account of the vicious cycle of philosophy scholarship reads convincing, it is unfortunately predicated on his previous argument that philosophy is really useless---nothing more than word salads. How did he reach that conclusion? He evaluated it more on its lack of utility, i.e., \enquote{the pursuit of useless knowledge.}

What's the solution? Graham's focus is on the test of utility:

\begin{quotation}
 The test of utility I propose is whether we cause people who read what we've written to do anything differently afterward. Knowing we have to give definite (if implicit) advice will keep us from straying beyond the resolution of the words we're using.
\end{quotation}

As we'll see later, Somers has two responses to why philosophy and the so-called word salads meet the test of utility.

Instead of looking for the most generalized principles regardless of their utility, like Aristotle, Graham works backwards. He instead looks within only the useful things, the most generalized principles. Unfortunately, this doesn't criticize philosophy meaningfully---his own test of utility seemed to have been met since Aristotle's times. He overlooked how ancient Greek political philosophy deeply influenced the foundation of democracy and the Western democratic tradition that we have today, and how philosophy about G-d\footnote{In Judaism, G-d is sometimes not fully spelled so as not to desecrate the name. As in Talmud (Shevuot 35a), \enquote{One who erases even a single letter of the Divine Name is liable.}} influenced Roman Catholicism. He himself concedes the influence of contemporary philosophy (e.g., controlled experiment) in the 20th century.

After all, did Graham really prove his ambitious claim? I'd say no. Graham didn't fulfill his fundamental burden of proving that philosophy is useless. In fact, history shows the philosophy has met the test of utility quite frequently ever since Aristotle.

Implicit in Graham's analysis is the natural evolution of a discipline. He praised Aristotle as the first people to think about abstract problems: \enquote{People talk so much about abstractions now that we don't realize what a leap it must have been when they [Aristotle et al.] first started to} (comment added). It is natural that a discipline in its infancy bears little value centuries later, especially with a subject as ambitious and generalized as philosophy. Graham's critique is just as apt for early philosophy as just about every discipline: early mathematics, early physics, early biology, etc. In fact, philosophy might even deserve a bit more time and patience as it's more extensive, generalized, and universal than the other disciplines.

\section{Somers for philosophy}

Somers started with listing the utility that Douglas Hofstadter, John Searle, etc., contribute to modern society. He then proceeded to defend the use of seemingly vague language. Even a seemingly trivial investigation on the word \enquote{I} can lead to interesting thoughts:
\begin{quotation}
 Thus, a petty semantic question—what is the meaning of the word \enquote{I}?—can open up a deeper line of inquiry: how do we form an integrated sense of self, why do we even have a word like “I,” when our brain is apparently just a soup of meaningless symbols? That is a question that, once articulated properly, can be explored by philosophers and scientists alike. (See Hofstadter’s “Who Shoves Whom Around Inside the Careenium?, or, The Meaning of the Word ‘I’” for one such exploration.)
\end{quotation}

\enquote{Word-wrestling} even has some practical utility in law, a fundamental element of our society:

\begin{quotation}
 [W]resting imprecise words out of their natural context and plugging them into a perverted formalism is the law as we know it—it’s practically the whole game. We don’t attack it, though, because we know that their “language-games” help to draw increasingly fine lines around difficult ideas—to, as it were, measure the coast of Britain. I think the same should be said of philosophy.
\end{quotation}

Further to arguing that pontificating upon complex words is useful for self-inquiry, Somers cited other examples to show that such word complexity is necessary. Much like the Stokes' Theorem, a mathematical theorem, a complicatedly worded sentence in philosophy embeds meaning that should be read beyond its \emph{prima facie} impression. People don't criticize the use of complicated words like \enquote{oriented smooth manifold} or \enquote{partition of unity} because they don't assume such terms are useless even if they don't understand them. In the same logic, one shouldn't assume that \emph{mutual historicization} or \emph{postmodern theory} are useless because they don't understand history and literature. The same logic for mathematics applies to philosophy.

\section{Conclusion}

In sum, it seems that Graham went too far with his claim that philosophy in the status quo is useless. It could instead be viewed as a natural progression that starts from being useless, much like how, before astronomy, there was astrology. Somers offered a good list of the usefulness of modern philosophy. Somers also explains the necessity of the so-called word salad, likening it to the use of complex mathematical terms.

Turns out a word salad is just as healthy as a green salad after all.

\reformcopyright{2025}

\end{document}